% !TeX program = lualatex

\input{../preamble.tex}

\begin{document}

% --- Title ---
\begin{center}
    {\fontspec{Segoe Script}
        \textcolor{bordercol}{\fontsize{36pt}{40pt}\selectfont Сказки Средневековья}
    }
\end{center}

\subsection*{Сказка о трёх медведях}

Жили-были три медведя. Жили они все вместе в лесу, в своем собственном доме. Один из них был маленький-малюсенький крошка-медвежонок, другой – средний медведь, а третий – большой-здоровенный медведище. У каждого был свой горшок для овсяной каши: у маленького-малюсенького крошки-медвежонка маленький горшочек, у среднего медведя средний горшок, у большого-здоровенного медведища большущий горшочище. Каждый медведь сидел в своем кресле: маленький-малюсенький крошка-медвежонок в маленьком креслице, средний медведь в среднем кресле, а большой-здоровенный медведище в большущем креслище. И спали они каждый на своей кровати: маленький-малюсенький крошка-медвежонок на маленькой кроватке, средний медведь на средней кровати, большой-здоровенный медведище на большущей кроватище.

Как-то раз сварили себе медведи на завтрак овсяную кашу, выложили ее в горшки, а сами пошли погулять по лесу: каше-то ведь простыть надо было; не то стали бы они ее есть горячую, она бы им весь рот обожгла.

А пока они гуляли по лесу, к дому подошла маленькая старушонка. Не очень-то она была хорошая, эта старушонка: сначала она заглянула в окошко, потом – в замочную скважину: увидела, что в доме никого нет, и подняла щеколду. Дверь была не заперта. Да медведи ее никогда не запирали – они были добрые медведи: сами никого не обижали и себе не ждали обиды.

Вот маленькая старушонка открыла дверь и вошла. И как же она обрадовалась, когда увидела на столе кашу! Будь она хорошей старушонкой, она, конечно, дождалась бы медведей, а те наверное угостили бы ее завтраком. Ведь они были хорошие медведи, грубоватые правда, как и все медведи, зато добродушные и гостеприимные. Но старушка была нехорошая, бессовестная и без спроса принялась за еду.

Сперва она попробовала каши из горшочища большого здоровенного медведища, но каша показалась ей слишком горячей, и старушонка сказала: «Дрянь!» Потом отведала каши из горшка среднего медведя, но его каша показалась ей совсем остывшей, и старушонка опять сказала: «Дрянь!» Тогда принялась она за кашу маленького-малюсенького крошки-медвежонка. Эта каша оказалась не горячей, не холодной, а в самый раз, и так понравилась маленькой старушонке, что она принялась уплетать ее за обе щеки и очистила весь горшочек до донышка. Однако oпротивная старушонка и эту кашу обозвала скверным словом: очень уж мал был горшочек, не хватило старушонке каши.

Потом старушонка села в креслище большого-здоровенного медведища, но оно показалось ей чересчур жестким. Она пересела в кресло среднего медведя, но оно показалось ей чересчур мягким. Наконец плюхнулась в креслице маленького-малюсенького крошки-медвежонка, и оно показалось ей не жестким, не мягким, а в самый раз. Вот уселась она в это креслице – сидела, сидела, пока не продавила сиденья и – шлеп! – прямо на пол. Поднялась противная старушонка и обозвала креслице бранным словом.

Тогда старушонка побежала наверх в спальню, где спали все три медведя. Сперва легла она на кроватищу большого-здоровенного медведища, но та показалась ей слишком высокой в головах. Потом легла на кровать среднего медведя, но эта показалась ей слишком высокой в ногах. Наконец легла на кроватку маленького-малюсенького крошки-медвежонка, и кроватка оказалась не слишком высокой ни в головах, ни в ногах, а – в самый раз. Вот укрылась старушонка потеплее и заснула крепким сном.

А к тому времени медведи решили, что каша уже поостыла, и вернулись домой завтракать. Глянул большой-здоровенный медведище на свой горшочище, видит, в каше ложка: там ее старушонка оставила. И взревел медведище своим громким грубым страшным голосом:

КТО-ТО МОЮ КАШУ ЕЛ!

Средний медведь тоже глянул на свой горшок, видит, и в его каше ложка.

Ложки-то у медведей были деревянные, – а будь они серебряные, противная старушонка наверняка бы их прикарманила.

И сказал средний медведь своим не громким, не тихим, а средним голосом:

КТО-ТО МОЮ КАШУ ЕЛ!

И маленький-малюсенький крошка-медвежонок глянул на свой горшочек, видит – в горшочке ложка, а каши и след простыл. И пропищал он тоненьким-тонюсеньким тихим голоском:

Кто-то мою кашу ел и всю ее съел!

Тут медведи догадались, что кто-то забрался к ним в дом и съел всю кашу маленького-малюсенького крошки-медвежонка. И принялись искать вора по всем углам. Вот большой-здоровенный медведище заметил, что твердая подушка криво лежит в его креслище – ее старушонка сдвинула, когда вскочила с места. И взревел большой-здоровенный медведище своим громким, грубым страшным голосом:

КТО-ТО В МОЕМ КРЕСЛИЩЕ СИДЕЛ!

Мягкую подушку среднего медведя старушонка примяла. И средний медведь сказал своим не громким, не тихим, а средним голосом:

КТО-ТО В МОЕМ КРЕСЛЕ СИДЕЛ!

А что сделала старушонка с креслицем, вы уже знаете. И пропищал маленький-малюсенький крошка-медвежонок своим тоненьким-тонюсеньким тихим голоском:

Кто-то в моем креслице сидел и сиденье продавил!

Надо искать дальше, решили медведи и поднялись наверх в спальню. Увидел большой-здоровенный медведище, что подушка его не на месте – ее старушонка сдвинула, – и взревел своим громким, грубым страшным голосом:

КТО-ТО НА МОЕЙ КРОВАТИЩЕ СПАЛ!

Увидел средний медведь, что валик его не на месте – это старушонка его передвинула, – и сказал своим не громким, не тихим, а средним голосом:

КТО-ТО НА МОЕЙ КРОВАТИ СПАЛ!

А маленький-малюсенький крошка-медвежонок подошел к своей кроватке, видит: валик на месте, подушка тоже на месте, а на подушке – безобразная, чумазая голова маленькой старушонки, и она-то уж никак не на месте: незачем было противной старушонке забираться к медведям!

И пропищал маленький-малюсенький крошка-медвежонок своим тоненьким-тонюсеньким тихим голоском:

Кто-то на моей кроватке спал и сейчас спит!

Маленькая старушонка слышала сквозь сон громкий, грубый страшный голос большого-здоровенного медведища, но спала так крепко, что ей почудилось, будто это ветер шумит или гром гремит. Слышала она и не громкий, не тихий, а средний голос среднего медведя, но ей почудилось, будто это кто-то во сне бормочет. А как услышала она тоненький-тонюсенький тихий голосок маленького-малюсенького крошки-медвежонка, до того звонкий, до того пронзительный, – сразу проснулась. Открыла глаза, видит – стоят у самой кровати три медведя. Она вскочила и бросилась к окну.

Окно было как раз открыто, – ведь наши три медведя, как и все хорошие, чистоплотные медведи, всегда проветривали спальню по утрам. Ну, маленькая старушонка и выпрыгнула вон; а уж свернула ли она себе шею, или заблудилась в лесу, или же выбралась из леса, но ее забрал констебль и отвел в исправительный дом за бродяжничество, – этого я не могу вам сказать. Только все три медведя никогда больше ее не видели.

\newpage

\subsection*{Вилли и поросёнок}

В благодарность за добрую услугу один прихожанин подарил молодому священнику из Данфермлина поросенка.
Сначала священник был в восторге от подарка, но поросенок быстро рос и прокормить его становилось все трудней. Вот священник и решил: «Пошлю-ка я его своему приятелю в Кэрнихилл. Пусть пасется там на воле: стоить это мне ведь ничего не будет».
А у священника был слуга, по имени Вилли, парень неплохой, но малость придурковатый.
- Вилли! - позвал его хозяин. - Сунь-ка поросенка в мешок и снеси в Кэрнихилл к моему приятелю, я уж с ним сговорился.
Но поросенок был хитрый, и Вилли пришлось повозиться, прежде чем он сумел поймать его и засунуть в мешок.
На дорогу священник сделал Вилли строгое напутствие. Он знал, что его верного слугу легко не стоит сбить с толку, из-за чего самое пустячное дело частенько оборачивалось для Вилли самым сложным. Так вот что он ему сказал:
- Смотри же, Вилли, нигде не проговорись, к кому ты идешь и зачем. Только сам помни тебе надо в Кэрнихилл. Отдашь там поросенка и тут же назад.
- Будьте спокойны, хозяин, - отвечал Вилли. - Вы же меня знаете! Все сделаю, как велите.
- То-то и оно-то, что я тебя хорошо знаю! - сказал священник.
И Вилли, взвалив драгоценную ношу на спину, отправился в путь. На полдороге к Кэрнихиллу он повстречал трех своих приятелей, окликнувших его с порога трактира.
- Привет, Вилли! - сказал один.
- Куда собрался в такой погожий денек, Вилли? - спросил второй.
- Что это ты тащишь в мешке, Вилли? - крикнул третий.
Вилли очень взволновала эта встреча.
- П-привет, друзья! - запинаясь, ответил он. - Я-я не могу вам сказать, куда я иду. Хозяин не велел мне говорить, куда я иду. А что у меня в мешке, я могу сказать: не кошка и не собака!
Друзья рассмеялись и тут же поспешили заверить Вилли, что не станут его ни о чем спрашивать. Один из них хлопнул Вилли по плечу и предложил:
- Заходи, Вилли, выпей с нами по стаканчику. Должно быть, ты устал с дальней дороги, да еще с тяжелой ношей на спине.
- Нет, мне нельзя, - отказался Вилли, бросая в то же время жадный взгляд на открытую дверь трактира, за которой, судя по всему, было так прохладно. - Уж коли хозяин доверил мне своего поросенка, пить мне никак нельзя!
Друзья перемигнулись, однако и виду не показали, что смекнули насчет поросенка.
- Да чего там, Вилли, заходи! Глоток вина никогда не повредит. А мешок оставь здесь у порога.
Больше уговаривать Вилли не пришлось. Он положил мешок с поросенком на землю и вошел в трактир. Тогда один из дружков, не теряя времени, развязал мешок, выпустил поросенка и посадил вместо него первую попавшуюся дворнягу.
Ничего не подозревающий честный слуга выпил стаканчик, взвалил опять на спину мешок и веселый пошел дальше. Добравшись до Кэрнихилла, он передал, как было велено, привет от своего хозяина его приятелю и вручил ему мешок.
- Спасибо тебе, Вилли, спасибо, - поблагодарил его приятель хозяина. - Не поможешь ли теперь развязать мешок и отвести поросенка в хлев?
Вилли развязал мешок, но вместо поросенка с розовым пятачком оттуда выскочила черная собачонка.
- Спасите! Спасите! - закричал бедный Вилли. - Не иначе сам дьявол сыграл со мной злую шутку.
Друг священника сильно удивился, однако не очень-то поверил насчет дьявольской шутки. А зная хорошо Вилли, подумал, что скорей всего кто-нибудь над ним по-дружески подшутил.
- Не стоит так волноваться, Вилли, - сказал он. - Можешь забрать свою собаку и отнести ее хозяину.
- Да это же не собака, сэр! - воскликнул Вилли, дрожа от страха. - Это поросенок! Клянусь вам - поросенок! Только дьявол поменял ему цвет: вместо белого сделал черным.
Но делать было нечего, и, засунув собаку в мешок, Вилли пустился в обратный путь. Добравшись до трактира, он опять увидел там трех своих приятелей, которые как ни в чем не бывало сидели за столом и с невинным видом потягивали вино.
- Никак, это Вилли! - сказал один. - И опять с мешком на спине?
- Ой-ой-ой, со мной случилось такое несчастье, - сказал, входя, Вилли. - Дьявол подменил моего поросенка собакой! Что я теперь скажу хозяину?
- Вот так штука! - воскликнул второй, едва сдерживаясь, чтобы не рассмеяться. - Вы слыхали что-нибудь подобное?
- Садись, дружище, тебе надо выпить после сильных переживаний, - сказал третий.
На этот раз Вилли совсем не пришлось уговаривать: ему действительно не мешало выпить. Он это заслужил.
Мешок он оставил на земле у входа в трактир и сел к столу. Ему налили - он выпил, а за это время один из шутников незаметно вышел и подменил собаку поросенком.
Через час-другой ничего не подозревающий Вилли уже шагал со своей ношей домой. В голове у него от выпивки и от всего пережитого царил полный кавардак. Он с ходу выложил свою жуткую историю хозяину, но тот так толком ничего и не понял.
- Не могу уразуметь, о чем ты говоришь, - сказал хозяин. - Какой дьявол? Какая собака? Развязывай скорей мешок и гони поросенка в хлев. Завтра понесешь его в Кэрнихилл.
- Да это же не поросенок, сэр! - воскликнул Вилли. - Это собака! Клянусь вам - собака! Вот, смотрите!
С этими словами Вилли развязал мешок, и оттуда выскочил поросенок. Вилли просто-таки завопил от ужаса:
- Как? Поросенок, а не собака?!
И бедный Вилли решил, что дьявол опять сыграл с ним злую шутку. А его хозяин…
А хозяин если раньше и сомневался в уме своего честного слуги, то теперь у него сомнений никаких не осталось.
А у вас?


\subsection*{Источники}

\begin{itemize}
    \item \href{https://sites.pitt.edu/~dash/folktexts.html}{sites.pitt.edu/~dash/folktexts.html} \\
\end{itemize}

\end{document}
